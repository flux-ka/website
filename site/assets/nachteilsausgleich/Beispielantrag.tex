% !TeX spellcheck = de_DE
\documentclass[a4paper, 11pt]{letter}
\usepackage[left=1in,top=0.2in,a4paper]{geometry}
\usepackage{url}
\usepackage[ngerman]{babel}
\usepackage[T1]{fontenc}
\usepackage[sfdefault]{biolinum}
\usepackage{todonotes}
\biolinum


\signature{Vorname Nachname (und natürlich unterschreiben $\hat{~}\hat{~}$ )}
\address{
	Vorname Nachname, Matrikelnummer
\\
Straße Hausnummer (Zimmernummer?)\\
PLZ Ort\\
E-Mail: \url{u****@student.kit.edu}\\
Tel: Telephonnummer
}

\longindentation=0pt

\begin{document}
\begin{letter}{
		Dein Studiengangservice\\
		z.Hd. Prüfungsausschuss \\
		Deine Fakultät des KIT\\
		\vspace{10mm}
		\textbf{Antrag auf Nachteilsausgleich für schriftliche Prüfungen}
	}
	\opening{Sehr geehrte Damen und Herren,}

	Ich studiere im $n$-ten Fachsemester <Studiengang> im <Bachelor|Master> und möchte aufgrund der Barrieren, die mir durch meine <Hier Lieblingsneurodivergenz einfügen> entstehen einen Nachteilsausgleich für schriftliche Prüfungen beantragen.\\
	\todo[inline]{Hier begründe ich jetzt wieso ich einen Nachteilsausgleich brauche.}
	Durch meine ASS und die daraus resultierende sensorische Hypersensibilität erzeugen Prüfungssituationen mit vielen Menschen und Umgebungsgeräuschen wie Klausuren einen hohen Stresspegel und sorgen für eine verringerte Konzentrationsfähigkeit. Es reicht in Klausuren beispielsweise aus, dass einer Person ein Stift herunterfällt, damit ich abgelenkt bin und eine lange Zeit benötige um meine Konzentrationsfähigkeit wiederzuerlangen. Allein durch die Anwesenheit vieler Menschen fällt es mir oft schwer in Klausuren die volle Leistung zu erbringen, was sich leider auch in meinen Noten widerspiegelt.\\
	\todo[inline]{Wieso helfen mir die im Attest genannten Maßnahmen gegen die Probleme? Es geht nicht darum zu begründen wieso es genau 30\% sein müssen, sondern vielmehr um das Allgemeine.}
	Diese Barrieren ließen sich vermeiden oder verringern, indem Ich, wie im Attest empfohlen, eine Schreibzeitverlängerung (20\%\textasciitilde 30\%) erhalte und in einem separaten Raum schreiben kann, um nicht abgelenkt zu werden. Alternativ wäre es auch möglich die Prüfung gegebenenfalls in mündlicher statt in schriftlicher Form abzulegen, da im Rahmen dieser keine Kommiliton:innen und generell weniger Personen anwesend wären.\\
	\todo[inline]{Ich habe meinen NTA erst im 5. Semester gestellt. Wenn man das spät macht ist es sinnvoll zu begründen weshalb, sonst denkt der Prüfungsausschuss, dass es ja bis jetzt auch ging.}
	Dass Ich den Antrag erst jetzt stelle, hat vor allem zwei Gründe:\\
	Zum einen ist die Diagnose noch sehr neu, da es leider mehrere Jahre dauern kann einen Diagnoseplatz zu erhalten und das gesamte Verfahren zu durchlaufen. Ich hatte zum Studienbeginn also noch keine feste Diagnose.\\
	Zum anderen war mir nicht bewusst, dass man als Studierender mit Behinderung/chronischen Krankheiten einen Nachteilsausgleich erhalten kann und Ich bin erst vor einem Semester durch Prof. Bellosa darauf aufmerksam geworden.\\
	Bei Rückfragen stehe ich unter den oben genannten Kontaktdaten gerne zur Verfügung.


	\closing{Mit freundlichen Grüßen}
	\todo[inline]{Das Attest muss dazu (also in die E-Mail), eine Diagnose geht den Prüfungsausschuss eigentlich nix an. Kann aber helfen sie beizulegen wenn man eine hat um dem ganzen mehr Nachdruck zu verleihen.}
	\encl{Psychiatrisches Attest und Diagnose}
\end{letter}
\end{document}